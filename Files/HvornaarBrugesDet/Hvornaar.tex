\section{Hvor når kunne man ønske at bruge cel shading/cartoon shading}

Hvorvidt toon shading er godt eller skidt er en meget subjektivt holdning og er i sidse end op til den enkelte, vi vil her se på to titler The Legend of Zelda: The Wind Waker og Bordelands som anvendte toon shading, hvorfor den beslutning blev taget og hvordan det blev modtaget. Med udgivelsen af en ny konsol og et skift fra at brug af kassetter til CD, havde Nintendo mulighede for at lave et Zelda spil som var lang mere fotorealistisk spil en før, det var med den tanke at de med annoncering af gamecube fremviste en demo\cite{ZeldaOnGameCube} som ledte fans og journalister til at tro det var denne retning spillet ville følge. På tros af den lang kraftigere konsol og mere lageplads, følte udvikler holdet at de sad fast og da af stand at til at finde på nye\cite{ZeldaOnGameCube} ideer,  den retning spillet havde tage føltes bare forkert. En dag valgte en af designerne at tage en tegneserieagtig \cite{ToonLinkIsBorn}link med ind på kontoret, dette fangede hele holdets opmærksomhed og det blev besluttet at spillet skulle gøre brug af cel shading. Denne beslutning betød også at udvikler holde kunne være mere fri med art stilen, problemet med fotorealistisk, i følge udvikleren egene mening, er at når man vil gøre spilleren opmærksom på noget i verden, være den en sti eller en del af et puslespil, så skal man få det til at stå ud, nogle gange så meget at det ikke længer passer i verden. En anden ting som udvikleren argumenterede for, var at man med denne teknik få art stillen til bedre at passe på tværs af plat formere og cover omslag \cite{ensArted}billider. Da spillet endeligt blev præsenteret for offentligheden mødets det med en kraftig modreaktion fra fans og journalister, som mente at det var sigtet mod \cite{EtSpilForBørn}børn. Da spillet endeligt kom ud mødets det med positiv \cite{BigScore}anmelder kritik og er blandt de bedst kritiseret Zelda spil, det endt dog med at skuffe forventning for salgs tal og brød aldrig igennem som med Ocarina of Time.
