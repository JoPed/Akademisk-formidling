% command til at lave et billede
% #1: placering af billede
% #2: width/scale
% #3: sti til billede
% #4: billede text
% #5: label
\newcommand{\pic}[5]
{
\begin{figure} [#1]
\includegraphics[width=#2]{#3}
\caption{#4}
\label{#5}
\end{figure}
}

%Ny command til figurer, hvor man kan wrap tekst omkring billede. 
%#1: angivelse af hvilken side billedet står i. 
% #2: width/scale
% #3: sti til billede
% #4: billede tekst
% #5: label for reference til billede.
\newcommand{\wrapfig}[5]
{
\begin{wrapfigure}{#1}{0.45\textwidth}
\vspace{-10pt}
\centering
\includegraphics[width=#2]{#3}
\caption{\textit{#4}}
\vspace{-13pt}
\label{#5}
\end{wrapfigure}
}


\newcommand{\matharray}[2]
{
\begin{eqnarray}
{#1}
\label{#2}
\end{eqnarray}
}

%Kommand til nummering af sectioner, så der ikke medtages kapitel nummer, hvis der ikke er et kapitel. 
\renewcommand{\thesection}{\arabic{section}}