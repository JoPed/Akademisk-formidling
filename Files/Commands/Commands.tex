% command til at lave et billede
% #1: placering af billede
% #2: width/scale
% #3: sti til billede
% #4: billede text
% #5: label
\newcommand{\pic}[5]
{
\begin{figure} [#1]
\includegraphics[width=#2]{#3}
\caption{#4}
\label{#5}
\end{figure}
}

%Ny command til figurer, hvor man kan wrap tekst omkring billede. 
% #1: angivelse af hvilken side billedet står i.
% #2: Hvor tæt teksten "må" komme på billedet.  
% #3: width/scale
% #4: sti til billede
% #5: billede tekst
% #6: label for reference til billede.
\newcommand{\wrapfig}[6]
{
\begin{wrapfigure}{#1}{#2\textwidth}
\centering
\includegraphics[width=#3]{#4}
\caption{\textit{#5}}
\label{#6}
\end{wrapfigure}
}

\newcommand{\eq}[6]
{
\begin{equation}
\abovedisplayskip=#1
\belowdisplayskip=#2
\abovedisplayshortskip=#3
\belowdisplayshortskip=#4
{#5}
\label{#6}
\end{equation}
}

%Kommand til nummering af sectioner, så der ikke medtages kapitel nummer, hvis der ikke er et kapitel. 
\renewcommand{\thesection}{\arabic{section}}