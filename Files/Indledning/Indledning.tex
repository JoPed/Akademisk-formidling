\section*{Indledning}
\thispagestyle{empty}
Toon shading eller også kendt som cel shading er en form for shader man kan bruge til at give en tegnefilms effekt. Der findes flere forskellige måde at lave og bruge toon shading, for eksemble bruger Bastion toon shading til at skabe en følelse af at det er håndtegnet, eller Okami der blandt andet bruger toon shading til at skabe en effekt af at det er malet med vandfarver og blæk.
\\
Så, hvis der findes flere former for toon shading, hvad er de og hvordan laves de? Og Hvorfor vil man i det hele taget bruge toon shading? Hvilke kvaliteter kan forbedres med de forskellige former for toon shading?
\\
Så, hvis der findes flere former for toon shading, hvad er de og hvordan laves de? Og Hvorfor vil man i det hele taget bruge toon shading? Hvilke kvaliteter kan forbedres med de forskellige former for toon shading?
På hvilke tekniske måder kan 3d figurer gøres "tegneserieagtig"
i spil og hvorfor kan det være ønskeligt?

\subsection*{Læsevejledning}
Kilder der er brugt i denne tekst, bliver angivet ved [nummer på kilde]. 
Nummeret i parentens indeholder en reference til en kilde fra kildelisten, med det første tal værernede nul. 
Yderligere skal bemærkes, at kilderne i kildelisten er sorteret efter rækkefølgen de optråde i denne tekst og ikke f.eks. titel, udgivelses år eller forfatter. 
Hvad angår fodnoter, bliver de brugt enkelte steder, til korte forklaringer hvis der bruges ikke danske udtryk. Det fremgår ved et tal løftet op, som i dette eksempel.\footnote[1]{Eksempel på fodnote}  

Alle billeder, som indgår i en del af teksten har som minimum et nummer, f.eks. figur 1. Billeder som bruges til, at forklare noget der er skrevet i teksten indeholder ikke
billede tekst, fordi de bliver bestrevet direkte i teksten. Fremtæder der tekst ved billeder, fremhæves dette ved kursiv skrift, det samme gør sig gældende for formler som
yderligere har tildelt et nummer startende fra et ved den første formel. Tabeller bliver på lige fod med billeder givet et nummer, f.eks. tabel 1, samtidig hvis nødvendigt en kort beskrivende tekst. Der bliver nogle steder i teksten refereret til specifikke kolonner i en bestemt tabel. Dette vil fremgå ved først, at angive tabellen og dernæst kolonnen. Vi angtager, at tabellens første kolonne er i venstre side som derfor bliver nummeret 1. 

Afsnit i teksten er tildelt et nummer, startende fra det for første afsnit, som bruges til, opdeling af indholdfortegnelse. 
