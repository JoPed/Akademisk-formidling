\section*{Indledning}
\thispagestyle{empty}
Toon shading eller også kendt som cel shading er en form for shader man kan bruge til at give en tegnefilms effekt. Der findes flere forskellige måde at lave og bruge toon shading, for eksemble bruger Bastion toon shading til at skabe en følelse af at det er håndtegnet, eller Okami der blandt andet bruger toon shading til at skabe en effekt af at det er malet med vandfarver og blæk.
\\
Så, hvis der findes flere former for toon shading, hvad er de og hvordan laves de? Og Hvorfor vil man i det hele taget bruge toon shading? Hvilke kvaliteter kan forbedres med de forskellige former for toon shading?
\\
\\
Så, hvis der findes flere former for toon shading, hvad er de og hvordan laves de? Og Hvorfor vil man i det hele taget bruge toon shading? Hvilke kvaliteter kan forbedres med de forskellige former for toon shading?
På hvilke tekniske måder kan 3d figurer gøres "tegneserieagtig"
i spil og hvorfor kan det være ønskeligt?

\subsection*{Læsevejledning}
Skriv noget om fodnoter og kildereferencer!!!

Alle billeder, som indgår i en del af teksten har som minimum et nummer, f.eks. figur 1. Billeder som bruges til, at forklare noget der er skrevet i teksten indeholder ikke billede tekst, fordi de bliver bestrevet direkte i teksten. 

