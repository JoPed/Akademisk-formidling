\newpage
\section{Introduktion}
\setcounter{page}{1}
Cel shading er en stilart af computer rendering, hvor man erstatter shadens skråning med flade farver og skygger. Med andre ord prøver man med cel shading af få det til at se ud som om det er tegnet i hånden. Det blev første gang set i 2000, da Sega udgav Jet Set Radio \cite{tvtropes2016}.
I artiklen \cite{Kinkley2016}, hvor Jonathan Kinkley med-direktør på Chicago Video Game Art Gallery, forklarer om cel shading i spil, beskrives ’The Legend Of Zelda: The Wind Waker, som værende det mest kendte spil hvor der er brugt cel shading. 

Cel shading er et eksempel på ikke fotorealistisk rendering, som betyder, at  det tydelig fremstå, at der med vilje er gjort en indsats for at lave computer grafik som ikke ligner fotografier eller fysiske objekter. Cel shading er alle dele af ikke fotorealistiske rendering, hvor man får grafikken til ligne tegneserie eller hånd tegnet. 
En måde at opnå denne effekt er, at opstege figurer med sorte streger. Eksempel på brugen af sorte streger kan ses i afsnit 2 'Cel shading', billede \ref{fig:jetsetradio2000}.