\newpage
\section{Introduktion}
\setcounter{page}{1}
Cel shading er en stilart af computer rendering, hvor man erstatter shadens skråning med flade farver og skygger. Med andre ord prøver man med cel shading af få det til at se ud som om det er tegnet i hånden. Det blev første gang set i 2000, da Sega udgav Jet Set Radio \cite{tvtropes2016}.
I artiklen \cite{Kinkley2016}, hvor Jonathan Kinkley med-direktør på Chicago Video Game Art Gallery, forklarer om cel shading i spil, beskrives ’The Legend Of Zelda: The Wind Waker, som værende det mest kendte spil hvor der er brugt cel shading.
\\
Cel shading kan gøre på flere forskellige måde, en af dem er at benytte sig af sorte streger rundt om figurer og lignende grafik. 