\newpage
\section{Introduktion}
\setcounter{page}{1}
Cel shading er en stilart af computer rendering, hvor man erstatter shadens skråning med flade farver og skygger. Med andre ord prøver man med cel shading af få det til at se ud som om grafikken ligner tegneserie grafik eller håndtegninger. Cel shading er en ikke fotorealistisk rendering som betyder, at grafikken med vilje bliver lavet med tegneserie eller den håndtegnede effekt, hvorved det bliver forskelligt fra fotografier eller fysiske objekter. \\Cel shading blev første gang brugt i spil i 2000, da Sega udgav  Smilebit's Jet Set Radio \cite{tvtropes}. I artiklen \cite{kinkley}, hvor Jonathan Kinkley med-direktør på Chicago Video Game Art Gallery, forklarer om cel shading i spil, beskrives ’The Legend Of Zelda: The Wind Waker, som værende det mest kendte spil hvor der er brugt cel shading, men samtidig et af de mindst populære Zelda spil.\\

En måde at opnå cel shading effekten på er, at opstege figurer med sorte streger. Eksempel på brugen af sorte streger kan ses i afsnit 2 'Cel shading', billede \ref{fig:jetsetradio2000}.