\newpage
\section{Alternativ toon shading}
Den bedst kendte metode til toon shading er cel shading, men der findes flere metoder til at opnå samme eller lidende effekt.\\
Man kan selvfølgelig tegne sin tekstur så den ligner tegneserie stilen, men lad os se bort fra det og kigge på de mere tekniske metoder. En alternativ metode kan lade sig gøre med et ’clone-push’ princip,\cite{clone-push} det er en metode hvor man laver en kopi af ens modeller gør dem helt sorte og en smugle stører, derefter ligger man den sorte kopi bag de gældende modeller, når computeren skal render modellerne får de et sort omrids som er typisk for tegneserie stilen.\\

\textbf{NPR} (non-photorealistic) teknikker kan også bruges til at opnå tegneserie stilen. I Simon M. Danner og Christoph J. Winklhofers rapport\cite{npr} beskriver Danner meget dybdegående toon rendering, Danner og Winklhofer skriver blandt andet om en NPR teknik som giver billeder en vandfarve effekt, som minder om kinesisk og japansk tegneserie stil. Denne NPR teknik form er brugt og fremstår tydeligt i video spil Okami:
\pic{h!}{25em}{Files/Alternativ_toon_shading/okami.png}{fremvisning af vandfarve brug i okami}{fig:okami}