\section{Cel shading}
Som beskrevet i afsnit 1 'Introduktion' opnås Cel shading ved f.eks. at optegne karaktere med sorte streger, et eksempel kan ses på figur \ref{fig:jetsetradio2000}, hvor den samme karakter er repræsentateret to gange. Henholdsvis med og uden brug af cel shading. Selv om karakterne er fra samme spil og derfor er lige gamle, virker karakteren til højre væsentlige nyere og meget pænere end karakteren til venstre. 
\pic{h}{15em}{Files/CelShading/jetsetradio.jpg}{Sammenligning af figur fra Jet Set Radio, hvor man tydeligt kan se de sorte streger på karakteren til højre}{fig:jetsetradio2000}
\\
Cel shading kan ikke kun opnås ved, at tegne på computeren. Man kan nemlig gøre det gennem programmering, i f.eks. OpenGL. Koden vil ikke blive vist og vil ikke blive forklaret, men den bagved liggende proces vil. Hjemmesiden Sunandblackcat \cite{sunandblackcat2016} har en beskrivelse af cel shading, samt en gennemgang af kode, som kan bruges til shading, beskivelse af processen laves med udgangspunkt af denne gennemgang. 

